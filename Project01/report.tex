\documentclass{article}
\usepackage{ctex}
\usepackage[fleqn]{amsmath}
\usepackage{amssymb}
\usepackage{bbm}
\usepackage[colorlinks]{hyperref}
\usepackage{graphicx}
\usepackage{float}

\graphicspath{{figs/}}
\renewcommand {\thefigure} {\arabic{section}.\arabic{figure}}

\title
{
	\normalfont\normalsize
	\textsc{数值分析与算法}\\ % Your university, school and/or department name(s)
	\vspace{7cm} % Whitespace
	{\huge 第一次大作业实验报告}\\ % The assignment title
	\vspace{6cm} % Whitespace
	\author{新雅62/CDIE6\quad 2016013327 项雨桐}
	\date{2019年11月12日} 
}

\begin{document}
\maketitle
\thispagestyle{empty}
\newpage
\thispagestyle{empty}
\tableofcontents
%---------------------
\newpage
\section{实验分析}
\setcounter{page}{1}
\setcounter{figure}{0}

\subsection{实验要求}
分别利用最近邻、双线性、双三次插值方法, 完成图像的旋转扭曲与畸变操作

\subsection{需求分析}
\setcounter{figure}{0}
图像操作的实现主要分为两个部分:
\begin{enumerate}
	\item 图像坐标变换:坐标变换是图像变换的核心操作。图像变换的基本范式是规定原图像$\mathrm{(srcImage)}$和目标图像$\mathrm{(dstImage)}$坐标间的函数关系,并通过数值计算求取对应的坐标值。
	\item 像素插值计算:通过$\mathrm{(srcImage)}$对应坐标的像素值,利用一定插值计算的方法获得$\mathrm{(dstImage)}$的对应坐标的像素值。对于RGB图像,可以对每个颜色通道分别插值后再组合获得新图像。
\end{enumerate} 

\section{方案设计与基本原理}
\setcounter{figure}{0}

编写完成旋转扭曲和畸变坐标变换的算法,通过$\mathrm{(dstImage)}$获得相应的$\mathrm{(srcImage)}$坐标,通过对$\mathrm{(srcImage)}$坐标进行插值计算,获得对应$\mathrm{(dstImage)}$坐标的像素值。

\subsection{坐标变换}
\subsubsection{旋转扭曲坐标变换}

旋转扭曲坐标变换需要设置两个参数:最大旋转角$\mathrm{\alpha_{max}}$和旋转半径$radius$。旋转中心为图像中央。

设$\mathrm{(dstImage)}$的某点坐标为$(x', y')$,则可得旋转角为

$$\alpha = \alpha_{max} \cdot \frac{radius-distance}{radius}$$

其中$distance$是$(x',y')$到图像中心的欧式距离。

则对应的$\mathrm{(srcImage)}$的坐标$(x,y)$满足:
\begin{equation}
\left\{
	\begin{aligned}
		x = x'\cos{\alpha} - y'\sin{\alpha}\\
		y = x'\sin{\alpha} + y'\cos{\alpha}\\
	\end{aligned}
\right.
\end{equation}

本次实验中选取的旋转角分别为$\frac{\pi}{4}$,$\frac{\pi}{2}$,$\frac{3\pi}{4}$,$\pi$;由于图像为正方形$(512\times512)$,因此选取的旋转半径为图像尺寸的一半(像素是离散的,中心点选为$(255.5,255.5)$,因此半径$radius = 255$)

\subsubsection{图像畸变坐标变换}

图像畸变坐标变换需要设置两个参数:正/负畸变和畸变半径$radius$,畸变中心为图像中央。

设$\mathrm{(dstImage)}$的某点坐标为$(x', y')$,则可得正畸变下$\mathrm{(srcImage)}$的坐标$(x,y)$满足:

\begin{equation}
	\left\{
		\begin{aligned}
			x = [\frac{radius}{distance}\arcsin{(\frac{distance}{radius})}] \cdot x'\\
			y = [\frac{radius}{distance}\arcsin{(\frac{distance}{radius})}] \cdot y'\\
		\end{aligned}
	\right.
\end{equation}

负畸变下$\mathrm{(srcImage)}$的坐标$(x,y)$满足:
\begin{equation}
	\left\{
		\begin{aligned}
			x = [\frac{radius}{distance}\sin{(\frac{distance}{radius})}] \cdot x'\\
			y = [\frac{radius}{distance}\sin{(\frac{distance}{radius})}] \cdot y'\\
		\end{aligned}
	\right.
\end{equation}

其中$distance$是$(x',y')$到图像中心的欧式距离。

本次实验中选取的畸变半径分别为$\frac{3}{4}radius$和$radius$,其中$radius$是旋转变换中使用的半径。

\subsection{插值算法}
设$(x',y')$经坐标变换得到的$\mathrm{(srcImage)}$浮点型坐标为$(x,y)$,令
\begin{equation}
	\left\{
		\begin{aligned}
			x = i + u\\
			y = j + v\\
		\end{aligned}
	\right.
\end{equation}
其中 $i = floor(x)$, $j = floor(y)$,因此 $ u,v \in [0,1] $

设$f(m,n)$代表$\mathrm{(srcImage)}$在坐标$(m,n)$处的像素值(单通道),则$\mathrm{(dstImage)}$在$\mathrm{(srcImage)}$对应坐标的像素值表示为$f(x,y)$

\subsubsection{最近邻插值}
\begin{equation}
	f(x,y)=
	\begin{cases}
		f(i,j) & u \leqslant 0.5, v \leqslant 0.5\\
		f(i + 1,j) & u > 0.5, v \leqslant 0.5\\
		f(i,j + 1) & u \leqslant 0.5, v > 0.5\\
		f(i + 1,j + 1) & u > 0.5, v > 0.5
	\end{cases}
\end{equation}

\subsubsection{双线性插值}
\begin{equation}
	f(x,y)=
	\begin{bmatrix} 
		1-u & u
	\end{bmatrix}
	\begin{bmatrix} 
		f(i,j) & f(i,j+1) \\
		f(i+1,j) & f(i+1,j+1)
	\end{bmatrix}
	\begin{bmatrix} 
		1-v \\ v
	\end{bmatrix}
\end{equation}

\subsubsection{双三次插值}

$$f(x,y) = ABC$$

其中
\begin{equation}
	A=
	\begin{bmatrix} 
		S(1+u) & S(u) & S(1-u) & S(2-u)
	\end{bmatrix}
\end{equation}

\begin{equation}
	B=
	\begin{bmatrix} 
		f(i-1,j-1) & f(i-1,j) & f(i-1,j+1) & f(i-1,j+2)\\
		f(i,j-1) & f(i,j) & f(i,j+1) & f(i,j+2)\\
		f(i+1,j-1) & f(i+1,j) & f(i+1,j+1) & f(i+1,j+2)\\
		f(i+2,j-1) & f(i+2,j) & f(i+2,j+1) & f(i+2,j+2)
	\end{bmatrix}
\end{equation}

\begin{equation}
	C=
	\begin{bmatrix} 
		S(1+v) \\ S(v) \\ S(1-v) \\ S(2-v)
	\end{bmatrix}
\end{equation}

插值基函数:
\begin{equation}
	S(x)=
	\begin{cases}
		1 - (a+3)|x|^2 + (a+2)|x|^3 & |x| \leqslant 1\\
		-4a + 8a|x| - 5a|x|^2 + a|x|^3 & 1 < |x| \leqslant 2\\
		0 & otherwise\\
	\end{cases}
\end{equation}

本实验中取$a = -1$

\section{误差分析}
\setcounter{figure}{0}

\noindent 本实验的误差主要包括以下四种:
\begin{itemize}
	\item 观测误差(采样误差):图像采集设备的光学、机械、电学特性给采样数据带来的误差
	\item 模型误差:由采样值重建原模拟图像与实际图像函数的误差
	\item 方法误差:用插值函数重建图像与理想值的误差
	\item 舍入误差:像素值的量化误差、浮点数据的存储与计算带来的误差等
\end{itemize}

\noindent 本实验的观测误差已不可考,下面分析其他三种误差:

\subsection{模型误差}
分析图像重建带来的误差,先引入图像采样与重建的概念。

\subsubsection{图像采样定理}
设图像在频域内是有限带宽的,也即:

$$\mathcal{F}(\Omega_x,\Omega_y) = 0$$

其中$\| \Omega_x \| > \frac{\pi}{\Delta x}, \| \Omega_y \| > \frac{\pi}{\Delta y}$,$\Delta x$、$\Delta y$是采样间隔,则利用采样值可以精确重建原图像。 设原图像坐标为$(x,y)$,对应的像素值为$f(x,y)$,则:

$$f(x,y) = \sum_{m=-\infty}^{+\infty}\sum_{n=-\infty}^{+\infty}f(m\Delta x,n\Delta y)\cdot \mathrm{sinc}(x-m\Delta x)\cdot \mathrm{sinc}(y-n\Delta y)$$

其中$\mathrm{sinc}(x) = \frac{\sin(\pi x)}{\pi x}$

在本实验中,我们取$\Delta x = \Delta y = 1$。

\subsubsection{误差分析}
若图像在频域内有限带宽,由IFT的知识可知图像在空域内是无限延展的;反之,对任意有限范围内的图像,其频域内都有无限的高频成分。但由于人眼对图像的高频成分不敏感,所以由频率截断(低通滤波)带来的误差可以忽略不计,也即认为图像采样定理一般是成立的。这样,在图像函数满足Dirichlet条件的前提下,利用$\mathrm{sinc(x)}$作为插值函数的图像重建函数与原图像的函数表达严格相等。

\subsubsection{总结}
在忽略低通滤波带来的截断误差、图像函数满足Dirichlet条件的前提下,图像函数

$$f(x,y) = \sum_{m=-\infty}^{+\infty}\sum_{n=-\infty}^{+\infty}f(m,n)\cdot \mathrm{sinc}(x-m)\cdot \mathrm{sinc}(y-n)$$

不存在模型误差。

\subsection{方法误差}
方法误差:

$$R(x,y) = f(x,y) - L(x,y)$$

其中$L(x,y)$是插值函数。

令
\begin{equation}
	\left\{
		\begin{aligned}
			x = i + u && i = \mathrm{floor}(x)\\
			y = j + v && j = \mathrm{floor}(y)
		\end{aligned}
	\right.
\end{equation}

\subsubsection{最近邻插值}
插值函数:

\begin{equation}
	\begin{aligned}
	L(x,y) = f(i,j)\cdot\mathbbm{1}\bigg|_{u\leqslant 0.5, v\leqslant 0.5} + f(i+1,j)\cdot\mathbbm{1}\bigg|_{u > 0.5, v\leqslant 0.5} \\ 
	+ f(i,j+1)\cdot\mathbbm{1}\bigg|_{u\leqslant 0.5, v > 0.5} + f(i+1,j+1)\cdot \mathbbm{1} \bigg|_{u > 0.5, v > 0.5}
	\end{aligned}
\end{equation}

误差函数:
\begin{equation}
	\begin{aligned}
		& \mathrm{R}(x,y) = f(i,j) \cdot [\mathrm{sinc}(u) \cdot \mathrm{sinc}(v) - \mathbbm{1}\bigg|_{u\leqslant 0.5, v\leqslant 0.5}] \\
		& + f(i+1,j) \cdot [\mathrm{sinc}(u-1) \cdot \mathrm{sinc}(v) - \mathbbm{1}\bigg|_{u > 0.5, v\leqslant 0.5}]\\ 
		& + f(i,j+1) \cdot [\mathrm{sinc}(u) \cdot \mathrm{sinc}(v-1) - \mathbbm{1}\bigg|_{u\leqslant 0.5, v > 0.5}]\\ 
		& + f(i+1,j+1) \cdot [\mathrm{sinc}(u-1) \cdot \mathrm{sinc}(v-1) - \mathbbm{1}\bigg|_{u > 0.5, v > 0.5}]\\ 
		& + \mathop{\sum_{m=-\infty}^{+\infty}\sum_{n=-\infty}^{+\infty}}_{\|x-m\|>1\atop \|y-n\|>1} f(m,n)\cdot \mathrm{sinc}(x-m)\cdot \mathrm{sinc}(y-n)\\
		& = \mathrm{Int}(u,v) + \mathrm{Res}(x,y)
	\end{aligned}
\end{equation}

$g(s,t) = \mathrm{sinc}(s)\cdot \mathrm{sinc}(t)$ 在 $[0,1]^2$ 上的图像如下所示:

\begin{figure}[H]
    \centering
    \includegraphics[width=\textwidth]{max_nearest.png}
    \caption{$\mathrm{sinc}(s)\cdot \mathrm{sinc}(t)$图像}
    \label{fig:3.1}
\end{figure}

因此当$s = t = 0.5$时$g(s,t)-\mathbbm{1}\bigg|_{s < 0.5, v < 0.5}$取最大值为$\frac{4}{\pi^2}$

则
\begin{equation}
	\begin{aligned}
		& \mathrm{Int}(u,v) \leqslant \frac{4}{\pi^2} \cdot \max\{f(i:i+1, j:j+1)\} \\
		& + (\frac{4}{\pi^2} - 1)\cdot(\sum_{k=i}^{i+1}\sum_{l=j}^{j+1} f(k,l)- \max\{f(i:i+1, j:j+1)\})\\
		& \approx 0.41 \cdot \max\{f(i:i+1, j:j+1)\} \\
		& - 0.59 \cdot(\sum_{k=i}^{i+1}\sum_{l=j}^{j+1} f(k,l)- \max\{f(i:i+1, j:j+1)\})
	\end{aligned}
\end{equation}

对于$\mathrm{sinc}(x-m)\cdot \mathrm{sinc}(y-n)$, 当$\|x-m\|>1, \|y-n\|>1$时,由于$x,y$独立,因此
$$\mathrm{sinc}(x-m)\cdot \mathrm{sinc}(y-n)\leqslant\mathrm{sinc^2}(w_m)$$
其中$w_{max}$是$\mathrm{sinc^2}(w)$的非零极值点,满足超越方程$\tan(w_{max}) = w_{max}$,$ \| w_{max} \| > 1 $,分析可知点列$w_{max}$的正数部分第一项为$w_1 = 1.43$,当$n\rightarrow \infty$时一致收敛于$n + 0.5$,因此$n + 0.4 < w_n < n + 0.5$。

故:
\begin{equation}
	\begin{aligned}
		& \mathrm{Res}(x,y) \leqslant \max{f(m,n)}\cdot\mathop{\sum_{m=-\infty}^{+\infty}\sum_{n=-\infty}^{+\infty}}_{\|x-m\|>1 \atop \|y-n\|>1} \mathrm{sinc}(x-m)\cdot \mathrm{sinc}(y-n) \\ 
		& \leqslant \max{f(m,n)}\cdot \sum_{w_{max}\in(-\infty,+\infty)\backslash\{0\}}\mathrm{sinc^2}(w_{max}) \\
		& \leqslant \max{f(m,n)}\cdot \sum_{w_{max}\in(-\infty,+\infty)\backslash\{0\}}[\frac{\sin(\pi w_{max})}{\pi w_{max}}]^2\nonumber\\
		& = \max{f(m,n)}\cdot \frac{2}{\pi^2}\sum_{w_{max}\in(0,+\infty)}[\frac{\sin(\pi w_{max})}{w_{max}}]^2\\
		& \leqslant \max{f(m,n)}\cdot \frac{2}{\pi^2}\sum_{w_{max}\in(0,+\infty)}(\frac{1}{w_{max}})^2\\
		& \leqslant \max{f(m,n)}\cdot \frac{2}{\pi^2}\sum_{n = 1}^{+\infty}(\frac{1}{n+0.4})^2\\
		& \leqslant \max{f(m,n)}\cdot \frac{2}{\pi^2}\cdot 1.03
	\end{aligned}
\end{equation}

因此
\begin{equation}
	\begin{aligned}
		& \mathrm{R}(x,y) \leqslant \frac{4}{\pi^2} \cdot \max\{f(i:i+1, j:j+1)\} \\
		& + (\frac{4}{\pi^2} - 1)\cdot(\sum_{k=i}^{i+1}\sum_{l=j}^{j+1} f(k,l)- \max\{f(i:i+1, j:j+1)\})\\
		& + \frac{2}{\pi^2}\sum_{n = 1}^{+\infty}(\frac{1}{n+0.4})^2 \cdot \max{f(m,n)}\\
		& \approx 0.41 \cdot \max\{f(i:i+1, j:j+1)\}\\ 
		& - 0.59 \cdot (\sum_{k=i}^{i+1}\sum_{l=j}^{j+1} f(k,l)- \max\{f(i:i+1, j:j+1)\})\\ 
		& + 0.21 \cdot \max_{m\in\mathbb{Z}\backslash\{i,i+1\}\atop n\in\mathbb{Z}\backslash\{j,j+1\}}{f(m,n)}
\end{aligned}
\end{equation}
是相关像素点值的线性组合

\subsubsection{双线性插值}
插值函数:
\begin{equation}
	\begin{aligned}
	& L(x,y) = f(i,j)\cdot (1-u)(1-v) + f(i+1,j)\cdot u(1-v) \\
	& + f(i,j+1)\cdot (1-u)v + f(i+1,j+1)\cdot uv
	\end{aligned}
\end{equation}

误差函数:
\begin{equation}
	\begin{aligned}
		& \mathrm{R}(x,y) = f(i,j) \cdot [ \mathrm{sinc}(u) \cdot \mathrm{sinc}(v) - (1-u)(1-v)] \\
		& + f(i+1,j) \cdot [ \mathrm{sinc}(u-1) \cdot \mathrm{sinc}(v) - u(1-v)]\\ 
		& + f(i,j+1) \cdot [ \mathrm{sinc}(u) \cdot \mathrm{sinc}(v-1) - (1-u)v]\\ 
		& + f(i+1,j+1) \cdot [ \mathrm{sinc}(u-1) \cdot \mathrm{sinc}(v-1) - uv]\\ 
		& + \mathop{\sum_{m=-\infty}^{+\infty}\sum_{n=-\infty}^{+\infty}}_{\|x-m\|>1\atop \|y-n\|>1} f(m,n)\cdot \mathrm{sinc}(x-m)\cdot \mathrm{sinc}(y-n)\\
		& = \mathrm{Int}(u,v) + \mathrm{Res}(x,y)
	\end{aligned}
\end{equation}

定义$g(s,t) =  \mathrm{sinc}(s) \cdot \mathrm{sinc}(t) - (1-s)(1-t)$, $(s,t)\in [0,1]^{2}$

$g(s,t)$ 在 $[0,1]^2$ 上的图像如下所示:

\begin{figure}[H]
    \centering
    \includegraphics[width=\textwidth]{max_bilinear.png}
    \caption{$g(s,t)$图像}
    \label{fig:3.2}
\end{figure}

$(s,t)\approx (0.27,0.27)$时,$\max{g(s,t)} \approx 0.25$

$\mathrm{Res}(x,y)$的分析同最近邻插值,因此:
\begin{equation}
	\begin{aligned}
		& \mathrm{R}(x,y) \leqslant 0.25 * \max\{f(i:i+1, j:j+1)\} + 0.21 \cdot \max_{m\in\mathbb{Z}\backslash\{i,i+1\}\atop n\in\mathbb{Z}\backslash\{j,j+1\}}{f(m,n)}
	\end{aligned}
\end{equation}

也是相关像素点值的线性组合

\subsubsection{双三次插值}

双三次插值的分析方法和前两者类似,因公式过于繁琐不再赘述。另一种比较好的估计方法是:
\begin{equation}
	\begin{aligned}
	& \left|R_{x}(x, y)\right| \leqslant \frac{1}{4 !} \max \left|\frac{\partial^{4} f}{\partial x^{4}}\right||(1+u)u(1-u)(2-u)|=\frac{3}{128} \max \left|\frac{\partial^{4} f}{\partial x^{4}}\right|\\
	& \left|R_{y}(x, y)\right| \leqslant \frac{1}{4 !} \max \left|\frac{\partial^{4} f}{\partial x^{4}}\right||(1+v)v(1-v)(2-v)|=\frac{3}{128} \max \left|\frac{\partial^{4} f}{\partial y^{4}}\right|
	\end{aligned}
\end{equation}

也即
\begin{equation}
	\begin{aligned}
		|R(x,y)|\leqslant \left|R_{x}(x, y)\right|+\left|R_{y}(x, y)\right|=\frac{3}{128}\left(\max \left|\frac{\partial^{4} f}{\partial x^{4}}\right|+\max \left|\frac{\partial^{4} f}{\partial y^{4}}\right|\right)
	\end{aligned}
\end{equation}


\subsection{舍入误差}
舍入误差的来源如下:

\subsubsection{量化误差}
量化误差主要是指像素值在量化为离散值时产生的误差。由于像素值的量化是向下取整,所以绝对误差为1,相对误差为1/256。

另外,在双三次插值出现overshooting和undershooting现象时采用clamp函数修正也会带来误差。

\subsubsection{浮点数存储/计算误差}
\begin{itemize}
	\item 程序中的$\mathrm{(dstImage)}$坐标值和$\mathrm{(srcImage)}$中心点坐标值、旋转半径不存在误差\footnote{中心点坐标值十进制为255.5、旋转半径0.75倍radius为191.25,在二进制用float存储时可以准确表示}
	\item 利用坐标计算的距离$\mathrm{distance}$和最大旋转角$\mathrm{\alpha_{max}}$在存储时存在截断误差
	\item 利用二者计算得到变换后$\mathrm{(srcImage)}$对应的浮点坐标$(x,y)$、插值计算中得到的插值基函数、像素值存在计算带来的误差累积和本身的存储误差
\end{itemize}

\subsubsection{截断误差}
float格式利用23个比特位表示小数部分,因此所有float类型变量的截断误差是$(\frac{1}{2})^{-24}$。

\subsubsection{旋转坐标的计算误差}
\noindent 由
$$\| \Delta A\| \leqslant \max{\|\frac{\partial f}{\partial x}\|}\cdot \|\Delta x\| + \max{\|\frac{\partial f}{\partial y}\|}\cdot \|\Delta y\|$$
和
$$\alpha = \alpha_{max} \cdot \frac{radius-distance}{radius}$$
$$\Delta \mathrm{distance} = \Delta \mathrm{\alpha_{max}} = (\frac{1}{2})^{-24}$$
知:
$$\frac{\partial}{\partial \alpha_{max}} = \frac{radius - distance}{radius}$$
$$\frac{\partial}{\partial (distance)} = -\frac{\alpha_{max}}{radius}$$
因此:
$$\| \Delta \alpha\| \leqslant \max{\|\frac{radius - distance}{radius}\|}\cdot (\frac{1}{2})^{-24} + \max{\|\frac{\alpha_{max}}{radius}\|}\cdot (\frac{1}{2})^{-24}$$
对应的$\mathrm{(srcImage)}$的坐标$(x,y)$满足:
\begin{equation}
\left\{
	\begin{aligned}
		x = x'\cos{\alpha} - y'\sin{\alpha}\\
		y = x'\sin{\alpha} + y'\cos{\alpha}\\
	\end{aligned}
\right.
\end{equation}
则:
$$\frac{\partial x}{\partial \alpha} = -x'\sin{\alpha} - y'\cos{\alpha}$$
$$\frac{\partial y}{\partial \alpha} = x'\cos{\alpha} - y'\sin{\alpha}$$
因此:
$$\| \Delta x\| \leqslant \max{\|-x'\sin{\alpha} - y'\cos{\alpha}\|}\cdot \| \Delta \alpha\|$$
$$\| \Delta y\| \leqslant \max{\|x'\sin{\alpha} + y'\cos{\alpha}\|}\cdot \| \Delta \alpha\|$$

\subsubsection{畸变坐标的计算误差}
\noindent 正畸变下$\mathrm{(srcImage)}$的坐标$(x,y)$满足:
\begin{equation}
	\left\{
		\begin{aligned}
			x = [\frac{radius}{distance}\arcsin{(\frac{distance}{radius})}] \cdot x'\\
			y = [\frac{radius}{distance}\arcsin{(\frac{distance}{radius})}] \cdot y'\\
		\end{aligned}
	\right.
\end{equation}
则
$$\frac{\partial x}{\partial distance} = [\frac{1}{distance\sqrt{1-(\frac{distance}{radius})^2}} - \frac{radius}{distance^2}\arcsin{(\frac{distance}{radius})}] \cdot x'$$
$$\frac{\partial y}{\partial distance} = [\frac{1}{distance\sqrt{1-(\frac{distance}{radius})^2}} - \frac{radius}{distance^2}\arcsin{(\frac{distance}{radius})}] \cdot y'$$
因此:
$$\| \Delta x\| \leqslant \max{\|[\frac{1}{distance\sqrt{1-(\frac{distance}{radius})^2}} - \frac{radius}{distance^2}\arcsin{(\frac{distance}{radius})}] \cdot x'\|}\cdot (\frac{1}{2})^{-24}$$
$$\| \Delta y\| \leqslant \max{\|[\frac{1}{distance\sqrt{1-(\frac{distance}{radius})^2}} - \frac{radius}{distance^2}\arcsin{(\frac{distance}{radius})}] \cdot x'\|}\cdot (\frac{1}{2})^{-24}$$

\noindent 负畸变下$\mathrm{(srcImage)}$的坐标$(x,y)$满足:
\begin{equation}
	\left\{
		\begin{aligned}
			x = [\frac{radius}{distance}\sin{(\frac{distance}{radius})}] \cdot x'\\
			y = [\frac{radius}{distance}\sin{(\frac{distance}{radius})}] \cdot y'\\
		\end{aligned}
	\right.
\end{equation}
则
$$\frac{\partial x}{\partial distance} = [\frac{1}{distance}\cos{\frac{distance}{radius}} - \frac{radius}{distance^2}\sin{(\frac{distance}{radius})}] \cdot x'$$
$$\frac{\partial y}{\partial distance} = [\frac{1}{distance}\cos{\frac{distance}{radius}} - \frac{radius}{distance^2}\sin{(\frac{distance}{radius})}] \cdot y'$$
因此:
$$\| \Delta x\| \leqslant \max{\|[\frac{1}{distance}\cos{\frac{distance}{radius}} - \frac{radius}{distance^2}\sin{(\frac{distance}{radius})}] \cdot x'\|}\cdot (\frac{1}{2})^{-24}$$
$$\| \Delta y\| \leqslant \max{\|[\frac{1}{distance}\cos{\frac{distance}{radius}} - \frac{radius}{distance^2}\sin{(\frac{distance}{radius})}] \cdot x'\|}\cdot (\frac{1}{2})^{-24}$$

\subsubsection{最近邻插值的计算误差}
最近邻插值算法:
\begin{equation}
	f(x,y)=
	\begin{cases}
		f(i,j) & u \leqslant 0.5, v \leqslant 0.5\\
		f(i + 1,j) & u > 0.5, v \leqslant 0.5\\
		f(i,j + 1) & u \leqslant 0.5, v > 0.5\\
		f(i + 1,j + 1) & u > 0.5, v > 0.5
	\end{cases}
\end{equation}

其中:
\begin{equation}
	\left\{
		\begin{aligned}
			x = i + u\\
			y = j + v\\
		\end{aligned}
	\right.
\end{equation}
$i = floor(x)$, $j = floor(y)$,因此$\Delta u = \Delta x$,$\Delta v = \Delta y$

除去像素值本身的量化误差,由上述公式可知,算法引入的误差即为$\Delta u = \Delta x$和$\Delta v = \Delta y$

\subsubsection{双线性插值的计算误差}
双线性插值算法:
\begin{equation}
	\begin{aligned}
	& f(x,y) = f(i,j)\cdot (1-u)(1-v) + f(i+1,j)\cdot u(1-v) \\
	& + f(i,j+1)\cdot (1-u)v + f(i+1,j+1)\cdot uv
	\end{aligned}
\end{equation}
\begin{equation}
	\left\{
		\begin{aligned}
			x = i + u\\
			y = j + v\\
		\end{aligned}
	\right.
\end{equation}
$i = floor(x)$, $j = floor(y)$,因此$\Delta u = \Delta x$,$\Delta v = \Delta y$

除去像素值本身的量化误差,由上述公式可知:
$$\frac{\partial f}{\partial u} = (1-v)[f(i+1,j)-f(i,j)] + v[f(i+1,j+1)-f(i,j+1)]$$
$$\frac{\partial f}{\partial v} = (1-u)[f(i,j+1)-f(i,j)] + u[f(i+1,j+1)-f(i+1,j)]$$
因此:
\begin{equation}
	\begin{aligned}
		\| \Delta f\| \leqslant \max_{(u,v)\in [0,1]^{2}}{\|(1-v)[f(i+1,j)-f(i,j)] + v[f(i+1,j+1)-f(i,j+1)]\|}\cdot \|\Delta x\| \\
		+ \max_{(u,v)\in [0,1]^{2}}{\|(1-u)[f(i,j+1)-f(i,j)] + u[f(i+1,j+1)-f(i+1,j)]\|}\cdot \|\Delta y\|
	\end{aligned}
\end{equation}

\subsubsection{双三次插值的计算误差}
\noindent 插值基函数:
\begin{equation}
	S(x)=
	\begin{cases}
		1 - 2|x|^2 + |x|^3 & |x| \leqslant 1\\
		4 - 8|x| + 5|x|^2 - |x|^3 & 1 < |x| \leqslant 2\\
		0 & otherwise\\
	\end{cases}
\end{equation}
插值函数:
$$f(x,y) = ABC$$
其中
\begin{equation}
	A=
	\begin{bmatrix} 
		S(1+u) & S(u) & S(1-u) & S(2-u)
	\end{bmatrix}
\end{equation}

\begin{equation}
	B=
	\begin{bmatrix} 
		f(i-1,j-1) & f(i-1,j) & f(i-1,j+1) & f(i-1,j+2)\\
		f(i,j-1) & f(i,j) & f(i,j+1) & f(i,j+2)\\
		f(i+1,j-1) & f(i+1,j) & f(i+1,j+1) & f(i+1,j+2)\\
		f(i+2,j-1) & f(i+2,j) & f(i+2,j+1) & f(i+2,j+2)
	\end{bmatrix}
\end{equation}

\begin{equation}
	C=
	\begin{bmatrix} 
		S(1+v) \\ S(v) \\ S(1-v) \\ S(2-v)
	\end{bmatrix}
\end{equation}
则
\begin{equation}
	\begin{aligned}
		\frac{\partial f}{\partial u} = \frac{\partial A}{\partial u}BC\\
		\frac{\partial f}{\partial v} = AB\frac{\partial C}{\partial v}
	\end{aligned}
\end{equation}
其中
\begin{equation}
	\frac{\partial A}{\partial u}=
	\begin{bmatrix} 
		\frac{\partial S(1+u)}{\partial u}  & \frac{\partial S(u)}{\partial u} & \frac{\partial S(1-u)}{\partial u} & \frac{\partial S(2-u)}{\partial u}
	\end{bmatrix}
\end{equation}
\begin{equation}
	\frac{\partial C}{\partial v}=
	\begin{bmatrix} 
		\frac{\partial S(1+v)}{\partial v} \\ \frac{\partial S(v)}{\partial v} \\ \frac{\partial S(1-v)}{\partial v} \\ \frac{\partial S(2-v)}{\partial v}
	\end{bmatrix}
\end{equation}

对$t\in[0,1]$
$$\frac{\partial S(1+t)}{\partial t} = -1 + 4t - 3t^2$$
$$\frac{\partial S(t)}{\partial t} = -4t + 3t^2$$
$$\frac{\partial S(1-t)}{\partial t} = 1 + 2t -3t^2$$
$$\frac{\partial S(2-t)}{\partial t} = -24 + 22t - 3t^2$$

因此:
\begin{equation}
	\begin{aligned}
		\| \Delta f\| \leqslant \max_{(u,v)\in [0,1]^{2}}{\|\frac{\partial A}{\partial u}BC\|}\cdot \|\Delta x\| 
		+ \max_{(u,v)\in [0,1]^{2}}{\|AB\frac{\partial C}{\partial v}\|}\cdot \|\Delta y\|
	\end{aligned}
\end{equation}

\section{实验总结}
\subsection{利用插值法进行图像变形的一般过程}
\begin{enumerate}
	\item 图片预处理,一般包括通道分离等
	\item 确定坐标变换函数,由目标图像对应整型坐标得到原图像浮点坐标
	\item 利用插值算法,从原图像整型坐标的像素值获得浮点坐标的像素值,也即目标图像对应点的像素值
	\item 图片终处理,例如通道合成,偏差修正等
\end{enumerate}

\subsection{线性插值方法的特性}
线性插值方法表现在预测点的像素值是原图像像素值的线性组合。不同的线性组合系数、原图像像素值的选取会得到不同的线性插值方法。

最近邻插值是一种零次插值,表现在只选取了原图像的1个采样点来预测对应点的像素;双线性插值是两个方向上的一次插值,因此利用了最近的4个点的像素,插值基函数是线性的;双三次插值是两个方向上的三次插值,利用了附近16个点的像素值,插值基函数为3次,可以通过选取不同的参数来调节。

插值基函数的特性\footnote{\url{https://dailc.github.io/2017/11/01/imageprocess_bicubicinterpolation.html}}:
\begin{itemize}
	\item $S(0) = 1$
	\item $S(n) = 0 , n \in \mathbb{Z}\backslash\{0\}$
	\item $S(x) = 0$, 当 $x$ 超过插值范围时
\end{itemize}

本实验中的插值基函数是对理想插值函数$sinc(x)$的逼近

这三种线性插值均采用多项式函数作为插值基函数,如果图像函数$f(x,y)$满足一定的光滑特性,则可以证明这些经插值后的图像一致收敛于原图像函数;且多项式阶次越高、选取的采样点越多,方法误差就越小,表现在图像上则是人工痕迹例如锯齿等现象越不明显。因此,从获得的图像中可以看出,双三次插值得到的图像最自然,双线性次之,最近邻最差。但多项式阶次和采样点数目的增长带来的是计算复杂度的增加,因此在实际应用中,应当根据所需的性能与计算开销之间进行平衡,合理选取插值方案。

从分析中也可以看出,线性插值方法的误差对于各采样点像素值也是线性的。

另外还需指出的是,双三次插值有可能造成overshooting或undershooting现象,也即某像素预测值超出了0~255的范围\footnote{\url{https://github.com/pytorch/pytorch/issues/21044}}。表现在图像上就是产生许多纯度高的彩色噪点\footnote{其实是undershooting产生的负值在强制转换为uchar类型时变为overshooting}。这时需要利用clamp函数来将插值结果限制为0~255的范围。

\setcounter{figure}{0}

\end{document}
